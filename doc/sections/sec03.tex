\section{Experiment}
\label{sec:three}

In the following section, we lay out the framework we built to evaluate the
models, as well as the latter's implementation and parameter selection.  In
addition, we briefly highlight the measures used to derive the test scores.

\subsection{Framework}

The framework is built using Python 3.5.2 and the standard PyData libraries
NumPy\furl{http://www.numpy.org/}, pandas\furl{http://pandas.pydata.org/} and
scikit-learn\furl{http://scikit-learn.org} (by \citet{sklearn:2011}).  The data
visualization routines utilize matplotlib\furl{http://matplotlib.org/} and
seaborn\furl{http://seaborn.pydata.org/}.

We start with a general outline for implementing models, objects that require a
\texttt{fit} and \texttt{predict} interface, as shown in Listing~\ref{lst:base}.
For this experiment, we implement linear and polynomial regression models, as
well as the $k$-Nearest Neighbors algorithm ourselves, and borrow the random
forest model from scikit-learn.

\lstset{
  caption={\texttt{fit} and \texttt{predict} methods for the baseline model.},
  label={lst:base}
}
\begin{lstlisting}
def fit(self, x, y):
    self._model = numpy.mean(y)

def predict(self, x):
    return numpy.full(
        x.shape[0],
        self._model
    )
\end{lstlisting}

We continue with implementing general means to import and transform the data for
regression and classification tasks, followed by creating facilities to score
each model either on the training or test data set.  For the former the model is
trained and evaluated using (optionally) different cross-validation (CV)
algorithms, with 10-fold CV as the default.  The final reported score is the
average of all scores computed by training a given model and measuring its
performance on each of the 10 splits.  Whereas testing a model is done using the
full training data for the fitting phase.  Scores are evaluated with the
provided test data sets.

As a final step, we build the pipeline to automate feature selection and model
evaluation, given a set of models, their respective hyper parameters and a
scoring function.  The entire implementation can be found on
GitHub\furl{https://github.com/cfricke/studious-bassoon}.

\subsection{Model Parameters}

As outlined in Section~\ref{sec:two}, we choose to evaluate several models that
require hyper parameters.  For the polynomial regression, we opt for degree two,
as higher degrees are too inefficient when considering all feature interactions
and the entire sample space; the required memory exceeds our availability.

We test $k$-NN with 1 and 5 neighbors, and chose 10 and 50 trees for two random
forest models as a proxy for marginal comparison and to keep the running time of
evaluating 50 different subsets of features manageable.  Neither random forest
models have a limit given for the maximum depth.  The performance of the random
forest model can be further improved by an increasing number of decision trees.
This improvement comes with a significant increase in training time.  For the
purpose of this experiment, the chosen parameters are sufficient to make
evaluate the performance for different subsets of top-$k$ features.

\subsection{Evaluation Measures}

To measure the accuracy of our models, we chose different scoring functions for
each of the tasks.  The mean squared error (MSE) is typically used to evaluate
the performance of regression models.  For classification we define \fmeasure{}
in the following equation as the harmonic mean of \precision{}
($tp / (tp + fp)$) and \recall{} ($tp / (tp + fn)$).

\vspace{-1em}
\begin{equation}\label{eq:fmeasure}
  \text{\fmeasure} = 2 \cdot \frac{\precision \cdot \recall} {\precision +
    \recall}
\end{equation}

We implement the MSE and the logarithmic
loss\furl{https://www.kaggle.com/wiki/LogarithmicLoss} functions, but do not
consider the latter for our final evaluation, as it is extremely punishing
towards false positives and false negatives, irrespective of the overall
precision and recall the model provides.  However, initial test series with the
logarithmic loss indicate an almost identical performance pattern, with the
exception of the baseline.  It should be noted that the logarithmic loss scoring
function requires as input the probability of a label, whereas the \fmeasure{}
expects the label itself.  Therefore, we evenly round the probabilities for all
accuracy measures.

%%% Local Variables:
%%% mode: latex
%%% TeX-master: "../main"
%%% End:
