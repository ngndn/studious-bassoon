\section{Results}
\label{sec:four}

Show off performance characteristics for each method using f-measure, MSE and
Log-Bernoulli Loss in a table, like this~\ref{tab:regperf}, for example.

\begin{table}[t]
  \caption{Regression performance comparison of baseline (1), linear regression
    (2) and polynomial regression of degree two (2) as measured using the mean
    squared error on the test data set.}
  % \centering\small
  % \renewcommand{\tabcolsep}{1pt}
  \newcolumntype{C}{>{\centering\arraybackslash}X}%
  % \newcolumntype{R}{>{\raggedleft\arraybackslash}X}
  \begin{tabularx}{\linewidth}{@{}cCc@{}}
    \toprule
    \bfseries Model & \bfseries Top-$k$ Features & \bfseries MSE \\
    \midrule
    (1) & --- &  12.60835 \\
    (2) &  10 &  0.047144 \\
    (3) &  13 &  0.032151 \\
    \bottomrule
  \end{tabularx}
\label{tab:regperf}
\end{table}

\begin{table}[t]
  \caption{Classification performance comparison of baseline (1), KNN with 1
    (2) and 5 neighbours (3), random forest with 10 (4) and 50 (5) decision
    trees as measured using the \fmeasure{} on the test data set.}
  % \centering\small
  % \renewcommand{\tabcolsep}{1pt}
  \newcolumntype{C}{>{\centering\arraybackslash}X}%
  % \newcolumntype{R}{>{\raggedleft\arraybackslash}X}
  \begin{tabularx}{\linewidth}{@{}cCc@{}}
    \toprule
    \bfseries Model & \bfseries Top-$k$ Features & \bfseries \fmeasure{} \\
    \midrule
    (1) & --- & 0.771499 \\
    (2) &   4 & 0.805405 \\
    (3) &   4 & 0.799732 \\
    (4) &   6 & 0.812203 \\
    (5) &   6 & 0.814915 \\
    \bottomrule
  \end{tabularx}
\label{tab:clsperf}
\end{table}

%%% Local Variables:
%%% mode: latex
%%% TeX-master: "../main"
%%% End:
